% This is a comment

\documentclass{article}

\usepackage{times}

\begin{document}

\title{A summary of Ambedkar's ``Who were the untouchables''}
\author{Chetan Vaity}
\date{\today}
\maketitle

\begin{abstract}
This is simply a short summary of Ambedkar's book on Untouchability\cite{ambedkar1}. Useful for people who do not have the patience to read it completely. I have tried to simply present Ambedkar's point of view here and whereever my opinion is expressed, I have specifically put it in a gray box.
\end{abstract}

\section{The book in concise form}
\subsection{Concept of defilement, impurity, contamination}
\subsection{Impurity in Hindu customs and scriptures}
\subsection{Broken men}
\subsection{Rice's Racial theory}
\subsection{Contempt for Buddhists}
\subsection{Beef eating as the root cause}
\subsection{Why no beef eating in Hindus}
\subsubsection{To go one up against Buddhists}
\subsection{Buddhist Bhikshus ate meat}
\subsection{Cow killing as mortal sin appears in Gupta period}
\subsection{Broken men again}
\subsection{Answers to possible objections}
\subsubsection{Did broken men really eat flesh of dead cow}
\subsection{Why did broken men not give up beef eating when Brahmins and non-Brahmins abandoned it}
\subsection{When did untouchability arise}

\subsection{Summary}
\begin{enumerate}
\item There is no racial difference between the Hindus and the Untouchables;
\item The distinction between the Hindus and Untouchables in its original form, before the advent of Untouchability, was the distinction between Tribesmen and Broken Men from alien Tribes. It is the Broken Men who subsequently came to be treated as Untouchables;
\item Just as Untouchability has no racial basis so also has it no occupational basis;
\item There are two roots from which Untouchability has sprung:
  \begin{enumerate}
  \item Contempt and hatred of the Broken Men as of Buddhists by the Brahmins:
  \item Continuation of beef-eating by the Broken Men after it had been given up by others.
  \end{enumerate}
\end{enumerate}

\section{My Comments}
\subsection{Interesting ideas}
\subsubsection{Instead of Broken men, why not call them bhikshus living outside villages}
Why invoke broken men concept at all?
\subsection{Weak points}

\bibliographystyle{plain}
\bibliography{../bib/history}

\end{document}
