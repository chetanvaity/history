\documentclass{article}

\usepackage{times}
\usepackage{chetanv-common}

\begin{document}

\title{A summary of Ambedkar's ``Who were the Untouchables''}
\author{Chetan Vaity}
\date{\today}
\maketitle
% No hyphenation please
\hyphenpenalty=10000
\exhyphenpenalty=10000

\section{Note}
This is simply a short summary of Ambedkar's book on Untouchability\cite{ambedkar1}. Useful for people who do not have the patience to read it completely. I have tried to simply present Ambedkar's point of view here - specifically, all the text in the quotation blocks are from the book. Although, I personally feel the thesis is not very coherent, it is still important as it presents very interesting ideas.

I have only read Ambedkar's book and not the other articles and books which he refers to - some of which have been mentioned in the bibliography.

% -----
\section{Concept of defilement, impurity, contamination}

The book starts off by examining the concept of untouchability and what is the crux of the idea.

\begin{shadequote}
It will be agreed on all hands that what underlies Untouchability is the notion of defilement, pollution, contamination and the ways and means of getting rid of that defilement.
\end{shadequote}

The author then delves into looking at various examples of defilement.

\begin{shadequote}
Primitive Man believed that defilement was caused by
\begin{enumerate}
\item the occurrences of certain events;
\item contact with certain things; and
\item contact with certain persons.
\end{enumerate}

Primitive Man also believed in the transmission of evil from one person to another. To him the danger of such transmission was peculiarly acute at particular times such as the performance of natural functions, eating, drinking, etc. Among the events the occurrence of which was held by Primitive Man as certain to cause defilement included the following:
\begin{enumerate}
\item Birth
\item Initiation
\item Puberty
\item Marriage
\item Cohabitation
\item Death
\end{enumerate}

Expectant mothers were regarded as impure and a source of defilement to others. The impurity of the mother extended to the child also. Initiation and puberty are stages which mark the introduction of the male and the female to full sexual and social life. They were required to observe seclusion, a special diet, frequent ablutions, use of pigment for the body and bodily mutilation such as circumcision.

To the Primitive Man the worst form of pollution was death. Not only the corpse, but the possession of the belongings of the deceased were regarded as infected with pollution. The widespread custom of placing implements, weapons, etc., in the grave along with the corpse indicates that their use by others was regarded as dangerous and unlucky.

Purificatory ceremonies. The sprinkling of water and the sprinkling of blood by the person defiled were enough to make him pure. Among purificatory rites were included changing of clothes, cutting hair, nail, etc., sweat-bath, fire, fumigation, burning of incense and fanning with the bough of a tree. But Primitive Society had another method of getting rid of impurity. This was to transfer it to another person. It was transferred to someone who was already taboo.
\end{shadequote}

% ----
\section{Impurity in Hindu customs and scriptures}
Manusmriti defines many different impurities and ways for purification. The author gives some examples around death and menstruation. Thus he argues, Manusmriti is not very different from most other primitive societies.

\begin{shadequote}
But there is another form of Untouchability observed by the Hindus which has not yet been set out. It is the hereditary Untouchability of certain communities.

Defilement as observed by the Primitive Society was of a temporary duration which arose during particular times such as the performance of natural functions, eating, drinking, etc. or a natural crisis in the life of the individual such as birth, death, menstruation, etc. After the period of defilement was over and after the purificatory ceremonies were performed the defilement vanished and the individual became pure and associable. But the impurity of the 50-60 millions of the Untouchables of India, quite unlike the impurity arising from birth, death, etc., is permanent. The Hindus who touch them and become polluted thereby can become pure by undergoing purificatory ceremonies. But there is nothing which can make the Untouchables pure. They are born impure, they are impure while they live, they die the death of the impure, and they give birth to children who are born with the stigma of Untouchability affixed to them. It is a case of permanent, hereditary stain which nothing can cleanse.
\end{shadequote}

The author then proceeds to list the various tribes which are classified as Untouchables according to ``the Orders-in-Council issued under the Government of India Act of 1935''. This report contains a ``Schedule'' or List. (This, incidently, is where the now ubiquitous terms ``Scheduled Castes and Scheduled Tribes'' comes from.)

The author then delves into the unique aspects of untouchability as practised by the Hindus. 

\begin{shadequote}
\item Why do the Untouchables live outside the village?
\item What made their impurity permanent, and ineradicable?
\end{shadequote}

The author begins by describing a nomadic primitive society where wealth was primarily cattle. This mode of living was changing with  the spread of settled agriculture - where wealth became land. The transition from a nomadic existence to a settled life was not sudden and there were (and still are) cases of contact between nomadic tribes and settled tribes.

% -----
\section{Broken men}
\begin{shadequote}
(There were) continuous raids and fights between nomadic tribes and settled tribes. It is the result of the continuous tribal warfare which was the normal life of the tribes in their primitive condition. In a tribal war it often happened that a tribe instead of being completely annihilated was defeated and routed. In many cases a defeated tribe became broken into bits. As a consequence of this there always existed in Primitive times a floating population consisting of groups of Broken tribesmen roaming in all directions.

Every individual in Primitive Society belonged to a tribe. Nay, he must belong to the tribe. Outside the tribe no individual had any existence. He could have none. Secondly tribal organisation being based on common blood and common kinship an individual born in one tribe could not join another tribe and become a member of it. The Broken Men had, therefore, to live as stray individuals. In Primitive Society where tribe was fighting against tribe a stray                                 collection of Broken Men was always in danger of being attacked. They did not know where to go for shelter. They did not know who would attack them and to whom they could go for protection. That is why shelter and protection became the problem of the Broken Men.

The foregoing summary of the evolution of Primitive Society shows that there was a time in the life of Primitive Society when there existed two groups one group consisting of Settled tribes faced with the problem of finding a body of men who would do the work of watch and ward against the raiders belonging to Nomadic tribes and the other group consisting of Broken Men from defeated tribes with the problem of finding patrons who would give them food and shelter.
The next question is: How did these two groups solve their problems? Although we have no written text of a contract coming down to us from antiquity we can say that the two struck a bargain whereby the Broken Men agreed to do the work of watch and ward for the Settled tribes and the Settled tribes agreed to give them food and shelter. Indeed, it would have been unnatural if such an arrangement had not been made between the two especially when the interest of the one required the co-operation of the other.
One difficulty, however, must have arisen in the completion of the bargain, that of shelter. Where were the Broken Men to live? In the midst of the settled community or outside the Settled community? In deciding this question two considerations must have played a decisive part. One consideration is that of blood relationship. The second consideration is that of strategy. According to Primitive notions only persons of the same tribe, i.e.. of the same blood, could live together. An alien could not be admitted inside the area occupied by the homesteads belonging to the tribe. The Broken men were aliens. They belonged to a tribe which was different from the Settled tribe. That being so, they could not be permitted to live in the midst of the Settled tribe. From the strategic point of view also it was desirable that these Broken men should live on the border of the village so as to meet the raids of the hostile tribes. Both these considerations were decisive in favour of placing their quarters outside the village.

The Untouchables were originally only Broken Men. It is because they were Broken Men that they lived outside the village.
\end{shadequote}

The author suggests possible ways of proving this: Study of totems of touchables and untouchables. If this points to them belonging to different tribes  it strengthens the claim.
\emph{Anta, Antyaja, Antyavasin}: Sanskrit terms referring people living outside the village. The author suggests that these terms refer to end of the village.

\begin{shadequote}
The Mahar community is a principle Untouchable community in Maharashtra. It is the single largest Untouchable community found in Maharashtra. The following facts showing the relations between the Mahars and the Touchable Hindus are worthy of note: (1) The Mahars are to be found in every village; (2) Every village in Maharashtra has a wall and the Mahars have their quarters outside the wall; (3) The Mahars by turn do the duty of watch and ward on behalf of the village; and (4) The Mahars claim 52 rights against the Hindu villagers. Among these 52 rights the most important are:
\begin{enumerate}
\item The right to collect food from the villagers;
\item The right to collect corn from each villager at the harvest season; and
\item The right to appropriate the dead animal belonging to the villagers.
\end{enumerate}

The evidence arising from the position of the Mahars is of course confined to Maharashtra. Whether similar cases are to be found in other parts of India has yet to be investigated.
The Mahars have a tradition that the 52 rights claimed by them against the villagers were given to them by the Muslim kings of Bedar. This can only mean that these rights were very ancient and that the kings of Bedar only confirmed them.

These facts although meagre do furnish some evidence in support of the theory that the Untouchables lived outside the village from the very beginning. They were not deported and made to live outside the village because they were declared Untouchables. They lived outside the village from the beginning because they were Broken Men who belonged to a tribe different from the one to which the Settled tribe belonged.
\end{shadequote}

The author then goes on to give reasons against Rice's race theory for untouchability\cite{stanleyrice1}. Also, against the occupational theory for untouchability. Lets not discuss these here as I am reasonably convinced against both these theories.

Next, the author presents his reasons of the origin of untouchability - ``Contempt for Buddhists'' and ``Beef eating''.

% -----
\section{Contempt for Buddhists}
\begin{shadequote}
The Brahmins shunned the Untouchables. They did not bring to light the fact that the Untouchables also shunned the Brahmins. Nonetheless, it is a fact. 
The fact was noticed by Abbe Dubois who says \cite{dubois1}:
``Even to this day a Pariah is not allowed to pass a Brahmin Street in a village, though nobody can prevent, or prevents, his approaching or passing by a Brahmin's house in towns. The Pariahs, on their part will under no circumstances, allow a Brahmin to pass through their paracherries (collection of Pariah huts) as they firmly believe it will lead to their ruin''.
Mr. Hemingsway, the Editor of the Gazetteer of the Tanjore District says:
``These castes (Parayan and Pallan or Chakkiliyan castes of Tanjore District) strongly object to the entrance of a Brahmin into their quarters believing that harm will result to them therefrom''.

This antipathy can be explained on one hypothesis. It is that the Broken Men were Buddhists. As such they did not revere the Brahmins, did not employ them as their priests and regarded them as impure. The Brahmin on the other hand disliked the Broken Men because they were Buddhists and preached against them contempt and hatred with the result that the Broken Men came to be regarded as Untouchables.

We have no direct evidence that the Broken Men were Buddhists. No evidence is as a matter of fact necessary when the majority of Hindus were Buddhists. We may take it that they were.

If we accept that the Broken Men were the followers of Buddhism and did not care to return to Brahmanism when it became triumphant over Buddhism as easily as other did, we have an explanation for both the questions. It explains why the Untouchables regard the Brahmins as inauspicious, do not employ them as their priest and do not even allow them to enter into their quarters. It also explains why the Broken Men came to be regarded as Untouchables. The Broken Men hated the Brahmins because the Brahmins were the enemies of Buddhism and the Brahmins imposed untouchability upon the Broken Men because they would not leave Buddhism. On this reasoning it is possible to conclude that one of the roots of untouchability lies in the hatred and contempt which the Brahmins created against those who were Buddhist.
Can the hatred between Buddhism and Brahmanism be taken to be the sole cause why Broken Men became Untouchables? Obviously, it cannot be. The hatred and contempt preached by the Brahmins was directed against Buddhists in general and not against the Broken Men in particular. Since untouchability stuck to Broken Men only, it is obvious that there was some additional circumstance which has played its part in fastening untouchability upon the Broken Men. 
\end{shadequote}

% -----
\section{Beef eating as the root cause}
\begin{shadequote}
The Census Returns show that the meat of the dead cow forms the chief item of food consumed by communities which are generally classified as untouchable communities. No Hindu community, however low, will touch cow's flesh. On the other hand, there is no community which is really an Untouchable community which has not something to do with the dead cow. Some eat her flesh, some remove the skin, some manufacture articles out of her skin and bones.

This new theory receives support from the Hindu Shastras. The Veda Vyas Smriti contains the following verse which specifies the communities which are included in the category of \emph{Antyajas} and the reasons why they were so included ``The Charmakars (Cobbler), the Bhatta (Soldier), the Bhilla, the Rajaka (washerman), the Puskara, the Nata (actor), the Vrata, the Meda, the Chandala, the Dasa, the Svapaka, and the Kolika- these are known as Antyajas as well as others who eat cow's flesh.''
Generally speaking the Smritikars never care to explain the why and the how of their dogmas. But this case is exception. For in this case, Veda Vyas does explain the cause of untouchability. The clause ``as well as others who eat cow's flesh'' is very important. It shows that the Smritikars knew that the origin of untouchability is to be found in the eating of beef.

The theory of beef-eating as the cause of untouchability also gives rise to many questions. Critics are sure to ask: What is the cause of the nausea which the Hindus have against beef-eating? Were the Hindus always opposed to beef-eating? If not, why did they develop such a nausea against it? Were the Untouchables given to beef-eating from the very start? Why did they not give up beef-eating when it was abandoned by the Hindus? Were the Untouchables always Untouchables? If there was a time when the Untouchables were not Untouchables even though they ate beef why should beef-eating give rise to Untouchability at a later-stage? If the Hindus were eating beef, when did they give it up? If Untouchability is a reflex of the nausea of the Hindus against beef-eating, how long after the Hindus had given up beef-eating did Untouchability come into being? These questions must be answered. Without an answer to these questions, the theory will remain under cloud. It will be considered as plausible but may not be accepted as conclusive.
\end{shadequote}

% -----
\section{Why no beef eating in Hindus}
The author gives some examples about how beef eating was prevalant in ancient India.

\begin{shadequote}
The killing of cow for the guest had grown to such an extent that the guest came to be called \emph{``Go-ghna''} which means the killer of the cow. To avoid this slaughter of the cows the \emph{Ashvateyana Grahya Sutra} (1.24.25) suggests that the cow should be let loose when the guest comes so as to escape the rule of etiquette.
Reference may be made to the ritual relating to disposal of the dead to counter the testimony of the \emph{Apastamba Dharma Sutra} - which has details about cows organs being used while cremation.
\end{shadequote}

\begin{shadequote}
The correct view is that the testimony of the \emph{Satapatha Brahmana} and the \emph{Apastamba Dharma Sutra} in so far as it supports the view that Hindus were against cow-killing and beef-eating, are merely exhortations against the excesses of cow-killing and not prohibitions against cow-killing. Indeed the exhortations prove that cow-killing and eating of beef had become a common practice. That notwithstanding these exhortations cow-killing and beef-eating continued. That most often they fell on deaf ears is proved by the conduct of Yajnavalkya, the great Rishi of the Aryans. The first passage quoted above from the Satapatha Brahmana was really addressed to Yajnavalkya as an exhortation. How did Yajnavalkya respond? After listening to the exhortation this is what Yajnavalkya said: ``I, for one, eat it, provided that it is tender''.
That the Hindus at one time did kill cows and did eat beef is proved abundantly by the description of the Yagnas given in the Buddhist Sutras which relate to periods much later than the Vedas and the Brahmanas. The scale on which the slaughter of cows and animals took place was collosal.
\end{shadequote}

The author provides references in Buddhist texts which describe cow sacrifices used in Yagnas.

\begin{shadequote}
For ordinary purposes the division of Hindus into two classes Mansahari and Shakahari may be enough. But it must be admitted that it is not exhaustive and does not take account of all the classes which exist in Hindu society. For an exhaustive classification, the class of Hindus called Mansahari shall have to be further divided into two sub-classes : (i) Those who eat flesh but do not eat cow's flesh; and (ii) Those who eat flesh including cow's flesh; In other words, on the basis of food taboos, Hindu society falls into three classes : (i) Those who are vegetarians; (ii) Those who eat flesh but do not eat cow's flesh; and (iii) Those who eat flesh including cow's flesh. Corresponding to this classification, we have in Hindu society three classes : (1) Brahmins; (2) Non-Brahmins; and (3) The Untouchables. This division though not in accord with the fourfold division of society called Chaturvarnya, yet it is in accord with facts as they exist. For, in the Brahmins we have a class which is vegetarian, in the non-Brahmins the class which eats flesh but does not eat cow's flesh and in the Untouchables a class which eats flesh including cow's flesh.

Why then did the non-Brahmins give up eating beef? There appears to be no apparent reason for this departure on their part. But there must be some reason behind it. The reason I like to suggest is that it was due to their desire to imitate the Brahmins that the non-Brahmins gave up beef-eating. This may be a novel theory but it is not an impossible theory. As the French author, Gabriel Tarde has explained that culture within a society spreads by imitation of the ways and manners of the superior classes by the inferior classes. This imitation is so regular in its flow that its working is as mechanical as the working of a natural law. Gabriel Tarde speaks of the laws of imitation. One of these laws is that the lower classes always imitate the higher classes. 
\end{shadequote}

The author provides details from \emph{Atreya Brahmana} which talks about details of rituals surrounding animal sacrifice and division of flesh between Brahmins etc. This establishes that Brahmins were non-vegetarians (probably beef eaters as well).
Some oblique references from Manusmriti are also provided about it not prohibiting beef eating specifically.

\begin{shadequote}
The clue to the worship of the cow is to be found in the struggle between Buddhism and Brahmanism and the means adopted by Brahmanism to establish its supremacy over Buddhism. 
\end{shadequote}

\subsection{To go one up against Buddhists}
\begin{shadequote}
Brahmanism was on the wane and if not on the wane, it was certainly on the defensive. As a result of the spread of Buddhism, the Brahmins had lost all power and prestige at the Royal Court and among the people. They were smarting under the defeat they had suffered at the hands of Buddhism and were making all possible efforts to regain their power and prestige. Buddhism had made so deep an impression on the minds of the masses and had taken such a hold of them that it was absolutely impossible for the Brahmins to fight the Buddhists except by accepting their ways and means and practising the Buddhist creed in its extreme form. After the death of Buddha his followers started setting up the images of the Buddha and building stupas. The Brahmins followed it. They, in their turn, built temples and installed in them images of Shiva, Vishnu and Ram and Krishna etc.,-all with the object of drawing away the crowd that was attracted by the image worship of Buddha. That is how temples and images which had no place in Brahmanism came into Hinduism. The Buddhists rejected the Brahmanic religion which consisted of Yagna and animal sacrifice, particularly of the cow. The objection to the sacrifice of the cow had taken a strong hold of the minds of the masses especially as they were an agricultural population and the cow was a very useful animal. The Brahmins in all probability had come to be hated as the killer of cows in the same way as the guest had come to be hated as Gognha, the killer of the cow by the householder, because whenever he came a cow had to be killed in his honour. That being the case, the Brahmins could do nothing to improve their position against the Buddhists except by giving up the Yagna as a form of worship and the sacrifice of the cow.
That the object of the Brahmins in giving up beef-eating was to snatch away from the Buddhist Bhikshus the supremacy they had acquired is evidenced by the adoption of vegetarianism by Brahmins. Why did the Brahmins become vegetarian? The answer is that without becoming vegetarian the Brahmins could not have recovered the ground they had lost to their rival namely Buddhism. In this connection it must be remembered that there was one aspect in which Brahmanism suffered in public esteem as compared to Buddhism. That was the practice of animal sacrifice which was the essence of Brahmanism and to which Buddhism was deadly opposed. That in an agricultural population there should be respect for Buddhism and revulsion against Brahmanism which involved slaughter of animals including cows and bullocks is only natural. What could the Brahmins do to recover the lost ground? To go one better than the Buddhist Bhikshus not only to give up meat-eating but to become vegetarians- which they did.
\end{shadequote}

% -----
\section{Buddhist Bhikshus ate meat}
\begin{shadequote}
In the time of Buddha there was in Vaisali a wealthy general named Siha who was a convert to Buddhism. He became a liberal supporter of the Brethren and kept them constantly supplied with good flesh-food. When it was noticed abroad that the Bhikshus were in the habit of eating such food specially provided for them, the Tirthikas made the practice a matter of angry reproach. Then the abstemious ascetic Brethren, learning this, reported the circumstances to the Master, who thereupon called the Brethren together. When they assembled, he announced to them the law that they were not to eat the flesh of any animal which they had seen put to death for them, or about which they had been told that it had been slain for them. But he permitted to the Brethern as ``pure'' (that is, lawful) food the flesh of animals the slaughter of which had not been seen by the Bhikshus, not heard of by them, and not suspected by them to have been on their account. In the Pali and Ssu-fen Vinaya it was after a breakfast given by Siha to the Buddha and some of the Brethren, for which the carcass of a large ox was procured that the Nirgianthas reviled the Bhikshus and Buddha instituted this new rule declaring fish and flesh ``pure'' in the three conditions. The animal food now permitted to the Bhikshus came to be known as the ``three pures'' or ``three pure kinds of flesh'', and it was tersely described as ``unseen, unheard, unsuspected'', or as the Chinese translations sometimes have it ``not seen, not heard nor suspected to be on my account''.

As the Buddhist Bhikshus did eat meat the Brahmins had no reason to give it up. Why then did the Brahmins give up meat-eating and become vegetarians? It was because they did not want to put themselves merely on the same footing in the eyes of the public as the Buddhist Bhikshus.
\end{shadequote}

% -----
\section{Cow killing as mortal sin appears in Gupta period}
\begin{shadequote}
We have got the incontrovertible evidence of inscriptions to show that early in the 5th century A. D. killing a cow was looked upon as an offence of the deepest turpitude, turpitude as deep as that involved in murdering a Brahman. We have thus a copper-plate inscription dated 465 A.D. and referring itself to the reign of Skandagupta of the Imperial Gupta dynasty. It registers a grant and ends with a verse saying : ``Whosoever will transgress this grant that has been assigned (shall become as guilty as) the slayer of a cow, the slayer of a spiritual preceptor (or) the slayer of a Brahman''. A still earlier record placing go-hatya on the same footing as brahma hatya is that of Chandragupta II, grandfather of Skandagupta just mentioned. It bears the Gupta date 93, which is equivalent to 412 A.D. It is engraved on the railing which surrounds the celebrated Buddhist stupa at Sanchi, in Central India. This also speaks of a benefaction made by an officer of Chandragupta and ends as follows:  ... ``Whosoever shall interfere with this arrangement ... he shall become invested with (the guilt of) the slaughter of a cow or of a Brahman, and with (the guilt of) the five anantarya" Here the object of this statement is to threaten the resumer of the grant, be he a Brahminist or a Buddhist, with the sins regarded as mortal by each community. The anantaryas are the five mahapatakas according to Buddhist theology. They are: matricide, patricide, killing an Arhat, shedding the blood of a Buddha, and causing a split among the priesthood. The mahapatakas with which a Brahminist is here threatened are only two : viz., the killing of a cow and the murdering of a Brahman. The latter is obviously a mahapataka as it is mentioned as such in all the Smritis, but the former has been specified only an upapataka by Apastamba, Manu, Yajnavalkya and so forth. But the very fact that it is here associated with brahma-hatya and both have been put on a par with the anantaryas of the Buddhists shows that in the beginning of the fifth century A.D., it was raised to the category of mahapatakas. Thus go-hatya must have come to be considered a mahapataka at least one century earlier, i.e., about the commencement of the fourth century A.D.

The question is why should a Hindu king have come forward to make a law against cow-killing, that is to say, against the Laws of Manu? The answer is that the Brahmins had to suspend or abrogate a requirement of their Vedic religion in order to overcome the supremacy of the Buddhist Bhikshus. If the analysis is correct then it is obvious that the worship of the cow is the result of the struggle between Buddhism and Brahminism. It was a means adopted by the Brahmins to regain their lost position.
\end{shadequote}

% -----
\section{Broken men again}
\begin{shadequote}
The stoppage of beef-eating by the Brahmins and the non-Brahmins and the continued use thereof by the Broken Men had produced a situation which was different from the old. This difference lay in the face that while in the old situation everybody ate beef, in the new -situation one section did not and another did. The difference was a glaring difference. Everybody could see it. It divided society as nothing else did before. All the same, this difference need not have given rise to such extreme division of society as is marked by Untouchability. It could have remained a social difference. There are many cases where different sections of the community differ in their foods. What one likes the other dislikes and yet this difference does not create a bar between the two.
There must therefore be some special reason why in India the difference between the Settled Community and the Broken Men in the matter of beef eating created a bar between the two. What can that be? The answer is that if beef-eating had remained a secular affair-a mere matter of individual taste-such a bar between those who ate beef and those who did not would not have arisen. Unfortunately beef-eating, instead of being treated as a purely secular        matter, was made a matter of religion. This happened because the Brahmins made the cow a sacred animal. This made beef-eating a sacrilege. The Broken Men being guilty of sacrilege necessarily became beyond the pale of society.
As has been said, the Brahmins made the cow a sacred animal. They did not stop to make a difference between a living cow and a dead cow. The cow was sacred, living or dead. Beef-eating was not merely a crime. If it was only a crime it would have involved nothing more than punishment. Beef-eating was made a sacrilege. Anyone who treated the cow as profane was guilty of sin and unfit for association. The Broken Men who continued to eat beef became guilty of sacrilege.
Once the cow became sacred and the Broken Men continued to eat beef, there was no other fate left for the Broken Men except to be treated unfit for association, i.e., as Untouchables.
\end{shadequote}

% -----
\section{Answers to possible objections}
\subsection{Did broken men really eat flesh of dead cow}
\begin{shadequote}
The answer to the first question is that even during the period when beef-eating was common to both, the Settled Tribesmen and the Broken Men, a system had grown up whereby the Settled Community ate fresh beef, while the Broken Men ate the flesh of the dead cow. We have no positive evidence to show that members of the Settled Community never ate the flesh of the dead cow. But we have negative evidence which shows that the dead cow had become an exclusive possession and perquisite of the Broken Men. The evidence consists of facts which relate to the Mahars of the Maharashtra to whom reference has already been made. As has already been pointed out, the Mahars of the Maharashtra claim the right to take the dead animal. This right they claim against every Hindu in the village. This means that no Hindu can eat the flesh of his own animal when it dies. He has to surrender it to the Mahar. This is merely another way of stating that when eating beef was a common practice the Mahars ate dead beef and the Hindus ate fresh beef. The only questions that arise are : Whether what is true of the present is true of the ancient past? Can this fact which is true of the Maharashtra be taken as typical of the arrangement between the Settled Tribes and the Broken Men throughout India.
In this connection reference may be made to the tradition current among the Mahars according to which they claim that they were given 52 rights against the Hindu villagers by the Muslim King of Bedar. Assuming that they were given by the King of Bedar, the King obviously did not create them for the first time. They must have been in existence from the ancient past. What the King did was merely to confirm them. This means that the practice of the Broken Men eating dead meat and the Settled Tribes eating fresh meat must have grown in the ancient past. That such an arrangement should grow up is certainly most natural. The Settled Community was a wealthy community with agriculture and cattle as means of livelihood. The Broken Men were a community of paupers with no means of livelihood and entirely dependent upon the Settled Community. The principal item of food for both was beef. It was possible for the Settled Community to kill an animal for food because it was possessed of cattle. The Broken Men could not for they had none. Would it be unnatural in these circumstances for the Settled Community to have agreed to give to the Broken Men its dead animals as part of their wages of watch and ward? Surely not. It can therefore be taken for granted that in the ancient past when both the Settled Community and Broken Men did eat beef the former ate fresh beef and the latter of the dead cow and that this system represented a universal state of affairs throughout India and was not confined to the Maharashtra alone.
\end{shadequote}

% -----
\subsection{Why did broken men not give up beef eating when Brahmins and non-Brahmins abandoned it}
\begin{shadequote}
The law made by the Gupta Emperors was intended to prevent those who killed cows. It did not apply to the Broken Men. For they did not kill the cow. They only ate the dead cow. Their conduct did not contravene the law against cow-killing. The practice of eating the flesh of the dead cow therefore was allowed to continue. Nor did their conduct contravene the doctrine of Ahimsa assuming that it has anything to do with the abandonment of beef-eating by the Brahmins and the non-Brahmins. Killing the cow was Himsa. But eating the dead cow was not. The Broken Men had therefore no cause for feeling qualms of conscience in continuing to eat the dead cow. Neither the law nor the doctrine of Himsa could interdict what they were doing, for what they were doing was neither contrary to law nor to the doctrine.
As to why they did not imitate the Brahmins and the non-Brahmins the answer is two fold. In the first place, imitation was too costly. They could not afford it. The flesh of the dead cow was their principal sustenance. Without it they would starve. In the second place, carrying the dead cow had become an obligaton though originally it was a privilege. As they could not escape carrying the dead cow they did not mind using the flesh as food in the manner in which they were doing previously.
\end{shadequote}

% -----
\section{When did untouchability arise}
\begin{shadequote}
Can we fix an approximate date for the birth of Untouchability? I think we can, if we take beef-eating, which is the root of Untouchability, as the point to start from. Taking the ban on beef-eating as a point to reconnoitre from, it follows that the date of the birth of Untouchability must be intimately connected with the ban on cow-killing and on eating beef. If we can answer when cow-killing became an offence and beef-eating became a sin, we can fix an approximate date for the birth of Untouchability. When did cow-killing become an offence? We know that Manu did not prohibit the eating of beef nor did he make cow-killing an offence. When did it become an offence? As has been shown by Dr. D. R. Bhandarkar, cow killing was made a capital offence by the Gupta kings some time in the 4th Century A.D.
We can, therefore, say with some confidence that Untouchability was born some time about 400 A.D. It is born out of the struggle for supremacy between Buddhism and Brahmanism which has so completely moulded the history of India and the study of which is so woefully neglected by students of Indian history.
\end{shadequote}

% -----
\section{Summary}
\begin{shadequote}
  \begin{enumerate}
  \item There is no racial difference between the Hindus and the Untouchables;
  \item The distinction between the Hindus and Untouchables in its original form, before the advent of Untouchability, was the distinction between Tribesmen and Broken Men from alien Tribes. It is the Broken Men who subsequently came to be treated as Untouchables;
  \item Just as Untouchability has no racial basis so also has it no occupational basis;
  \item There are two roots from which Untouchability has sprung:
    \begin{enumerate}
    \item Contempt and hatred of the Broken Men as of Buddhists by the Brahmins:
    \item Continuation of beef-eating by the Broken Men after it had been given up by others.
    \end{enumerate}
  \end{enumerate}
\end{shadequote}

% -----
\bibliographystyle{alpha}
\bibliography{../bib/history}

\end{document}
